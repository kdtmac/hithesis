% !Mode:: "TeX:UTF-8"

\hitsetup{
  %******************************
  % 注意:
  %   1. 配置里面不要出现空行
  %   2. 不需要的配置信息可以删除
  %******************************
  %
  %=====
  % 秘级
  %=====
  statesecrets={公开},
  natclassifiedindex={TM301.2},
  intclassifiedindex={62-5},
  %
  %=========
  % 中文信息
  %=========
  % ctitleone={局部多孔质气体静压},%本科生封面使用
  % ctitletwo={轴承关键技术的研究},%本科生封面使用
  ctitlecover={基于深度学习的红外图像无人机检测},%放在封面中使用,自由断行
  ctitle={基于深度学习的红外图像无人机检测},%放在原创性声明中使用
  % csubtitle={一条副标题}, %一般情况没有,可以注释掉
  cxueke={工学},
  csubject={信息与通信工程},
  caffil={电子与信息工程学院},
  cauthor={陈逸飞},
  csupervisor={吴少川教授},
  % cassosupervisor={某某某教授}, % 副指导老师
  % ccosupervisor={某某某教授}, % 联合指导老师
  % 日期自动使用当前时间,若需指定按如下方式修改:
  cdate={2022年6月},
  cstudentid={20S005075},
  cstudenttype={学术学位论文}, %非全日制教育申请学位者
  % cnumber={no9527}, %编号
  % cpositionname={哈铁西站}, %博士后站名称
  % cfinishdate={20XX年X月---20XX年X月}, %到站日期
  % csubmitdate={20XX年X月}, %出站日期
  % cstartdate={3050年9月10日}, %到站日期
  % cenddate={3090年10月10日}, %出站日期
  %(同等学力人员)、(工程硕士)、(工商管理硕士)、
  %(高级管理人员工商管理硕士)、(公共管理硕士)、(中职教师)、(高校教师)等
  %
  %
  %=========
  % 英文信息
  %=========
  etitle={UAV target detection in Infrared image},
  % esubtitle={This is the sub title},
  exueke={Engineering},
  esubject={Information and Communication Engineering},
  eaffil={\emultiline[t]{School of Electronic and Information \\  Engineering}},
  eauthor={Chen YiFei},
  esupervisor={Professor Wu Shaochuan},
  % eassosupervisor={XXX},
  % 日期自动生成,若需指定按如下方式修改:
  edate={June, 2022},
  estudenttype={Master of Art},
  %
  % 关键词用“英文逗号”分割
  ckeywords={红外目标检测, YOLOv5, 轻量化},
  ekeywords={Infrared object detection, YOLOv5, Model compression},
}

\begin{cabstract}
  目标检测是对输入图像进行相应处理后得出图像中待测目标的位置和类别的一种任务。近年来随着无人机市场和技术的发展,对无人机目标的检测和打击成为了安全领域的重要任务。而红外成像相对于可见光成像有受天气影响小等优势,因此红外图像的目标检测是目标检测中的一个重要方向,针对这一问题,本文研究提出了一种基于深度学习的红外图像无人机目标检测算法。

本文研究分析了红外图像无人机检测任务中的几个难点,如缺少公开数据集、图像信噪比低、背景复杂、已有算法定位不准、嵌入式运行效率低等,提出了具有针对性的解决方案。首先本文选择了目标检测任务中表现较好的深度学习算法YOLOv5并且和其他常用算法进行了对比,实验结果表明YOLOv5算法在红外无人机检测任务中性能优于其他已有算法。针对红外图像开源数据集较少的问题,本文自行制作了红外无人机目标数据集。针对红外图像目标边缘模糊、分辨率低、信息量小等特点,本文设计了一种通道填充的数据增强算法,提升了算法的性能。针对无人机检测任务中背景复杂的问题,本文设计了一种图像拼接的数据增强算法,提升了算法对小目标的检测能力,从而提升了算法在红外无人机数据集上的检测精度。针对小目标较多造成的算法定位不准的问题,本文改进了YOLOv5算法的目标框定位损失函数,提升了算法在红外无人机数据集上的检测精度。针对实际检测任务中的实时处理需求,本文设计了一种轻量化算法并且在嵌入式设备上进行了验证,结果表明本文提出的轻量化红外无人机目标检测算法能在嵌入式设备上基本上达到实时处理的要求。

综上,本文针对红外图像的无人机目标检测问题,研究提出了一种基于深度学习的红外无人机目标检测算法,在红外无人机数据集上相对已有算法能取得更高的检测精度,该算法的轻量化版本在嵌入式设备上验证结果表示,算法运行速度达到每秒28帧,基本可以实现实时运行。
\end{cabstract}

\begin{eabstract}
  Object detection is a task of obtaining the position and category of the object to be detected in the image after corresponding processing of the input image. In recent years, the detection and attack of UAV targets has become an important task in the security field. Compared with visible light imaging, infrared imaging has the advantage of being less affected by the weather. Therefore, the target detection of infrared images is an important direction in target detection. In response to this problem, this paper proposes a deep learning-based infrared image UAV. Object detection algorithm.

This paper studies and analyzes several difficulties in the infrared image UAV detection task, such as lack of public data sets, low image signal-to-noise ratio, complex background, inaccurate positioning of existing algorithms, low embedded operation efficiency, etc. Sexual solution. First of all, this paper selects the deep learning algorithm YOLOv5, which performs better in the target detection task, and compares it with other commonly used algorithms. The experimental results show that the performance of the YOLOv5 algorithm in the infrared UAV detection task is better than other existing algorithms. Aiming at the problem that there are few open source datasets of infrared images, this paper produced an infrared UAV target dataset by itself. Aiming at the characteristics of infrared image target edge blur, low resolution and small amount of information, this paper designs a data enhancement algorithm for channel filling. Aiming at the problem of complex background in the UAV detection task, this paper designs a data enhancement algorithm for image stitching, which improves the algorithm's ability to detect small targets, thereby improving the detection accuracy of the algorithm on the infrared UAV dataset. Aiming at the problem of inaccurate positioning of the algorithm caused by many small targets, this paper improves the target frame positioning loss function of the YOLOv5 algorithm, and improves the detection accuracy of the algorithm on the infrared UAV dataset. Aiming at the real-time processing requirements in actual detection tasks, this paper designs a lightweight algorithm and validates it on embedded devices. The results show that the proposed lightweight infrared UAV target detection algorithm can basically be used on embedded devices. meet real-time processing requirements.

In summary, this paper proposes an infrared UAV target detection algorithm based on deep learning, which can achieve higher detection accuracy than existing algorithms on the infrared UAV dataset. The lightweight version of the algorithm is used in embedded devices. The above verification results show that the algorithm runs at a speed of 28 frames per second, which can basically achieve real-time operation.
\end{eabstract}
