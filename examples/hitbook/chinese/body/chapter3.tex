% !Mode:: "TeX:UTF-8"

\chapter{红外无人机目标检测算法}[Example]

\section{引言}
本课题研究的红外无人机目标检测问题,主要需要解决的问题有两个。
一个是原始图像是由红外设备产生的,因此原始图片是灰度图,
不含有普通BGR图像的三个通道信息;
另一个问题是在实际的检测场景下,无人机目标在检测设备中往往以小目标形式出现,
本课题对算法小目标的检测能力要求较高。
因此本章所做的研究首先是针对本课题的红外无人机目标检测问题
建立研究必要的数据集,
之后选择一种在该数据集上性能较好的深度学习算法
作为基础算法,接着针对上述的两个问题提出了
基于通道填充和图像合成的数据增强算法,
在mAP和推理时间两个方面
进一步提升算法在数据集上的性能表现。

\section{数据集制作}[Number]
由于神经网络需要大量的数据进行训练,
并且算法还需要在数据集上进行测试,
因此本课题研究需要一个红外无人机目标的数据集。
由于目前尚无开源红外无人机数据集可供使用,
本课题自建了红外无人机数据集。

\subsection{数据集概况}
目前国内外完善标注的公开红外数据集很少,
仅有少量红外无人机数据集但多以点标注形式发布,
因此需要自行收集红外无人机图像并进行标注。
考虑到网络模型的泛化能力和特征提取的鲁棒性,
自行收集了多种背景的红外图像,
并且覆盖距离远近的无人机目标,目标全部为旋翼无人机。
数据集来源主要是网络获取和实验室自主采集两种方式,
格式为红外无人机图像和视频,通过脚本转换为图片格式后,
标注目标框,制作成数据集,总共包含10000张图片,原始格式为PASCAL VOC,
按训练、验证、测试分别有7000张、2000张、1000张。
数据集中的原图示例如图\ref{dataset}所示。

\begin{figure}[!h]
    \setlength{\subfigcapskip}{-1bp}

    \centering
    \begin{minipage}{\textwidth}
    \centering
    \subfigure{\label{dataset11}}\addtocounter{subfigure}{-2}
    \subfigure{\subfigure[正常目标示例~1]{\includegraphics[width=0.4\textwidth]{1.png}}}
    \hspace{2em}
    \subfigure{\label{dataset12}}\addtocounter{subfigure}{-2}
    \subfigure{\subfigure[正常目标示例~2]{\includegraphics[width=0.4\textwidth]{2.png}}}
    \end{minipage}

    \centering
    \begin{minipage}{\textwidth}
    \centering
    \subfigure{\label{dataset21}}\addtocounter{subfigure}{-2}
    \subfigure{\subfigure[小目标示例~1]{\includegraphics[width=0.4\textwidth]{5.png}}}
    \hspace{2em}
    \subfigure{\label{dataset22}}\addtocounter{subfigure}{-2}
    \subfigure{\subfigure[小目标示例~2]{\includegraphics[width=0.4\textwidth]{6.png}}}
    \end{minipage}

    \centering
    \begin{minipage}{\textwidth}
    \centering
    \subfigure{\label{dataset31}}\addtocounter{subfigure}{-2}
    \subfigure{\subfigure[遮挡目标示例~1]{\includegraphics[width=0.4\textwidth]{3.png}}}
    \hspace{2em}
    \subfigure{\label{dataset32}}\addtocounter{subfigure}{-2}
    \subfigure{\subfigure[遮挡目标示例~2]{\includegraphics[width=0.4\textwidth]{4.png}}}
    \end{minipage}

    \vspace{0.2em}
    \caption{数据集图像示例}
    {\label{dataset}}
\end{figure}


\section{深度学习网络模型基本算法YOLOv5}
本课题算法的主干网络模型采用YOLOv5网络,网络的结构如图\ref{yolo1}所示,
该模型按照推理流程顺序主要可以分成
输入端、Backbone、Neck、Prediction这四个部分。

\begin{figure}[h]
    \centering
    \includegraphics[width = 0.8\textwidth]{yolov5结构.jpg}
    \caption{yolov5网络结构}
    \label{yolo1}
\end{figure}

\subsection{输入端}
YOLOv5网络的输入端就是输入图像的模块,
既可以向网络提供原始的图像输入,
也可以在这个部分对图像进行预处理
和数据增强,
此外还能进行算法的准备工作,比如锚框的
自适应计算等。

\subsection{Backbone}
Backbone的主要作用是初步提取特征图。其中典型的模块是Focus模块、CBL模块以及CSP模块。

\subsubsection{Focus模块}
Focus模块在YOLOv5中是图片进入backbone前,对图片进行切片操作,
具体操作是在一张图片中每隔一个像素拿到一个值,类似于邻近下采样,
这样就拿到了四张图片,四张图片互补,长的差不多,但是没有信息丢失,
这样一来,将W、H信息就集中到了通道空间,输入通道扩充了4倍,
即拼接起来的图片相对于原先的RGB三通道模式变成了12个通道,
最后将得到的新图片再经过卷积操作,最终得到了没有信息丢失情况下的二倍下采样特征图。

\begin{figure}[h]
  \centering
  \includegraphics[width = 0.8\textwidth]{focus.png}
  \caption{Focus切片示意图}
  \label{focus}
\end{figure}

\subsubsection{CBL模块}
CBL模块是卷积神经网络的基本组成模式,
结构如图\ref{cbl}所示,
由卷积、归一化、激活函数三个部分组成。
CBL的主要作用是通过卷积采集特征。
其中的激活函数默认是Leaky ReLU,Leaky Rectified Linear Unit是一种基于 ReLU 的激活函数,但它对于负值的斜率很小,而不是平缓的斜率。 斜率系数是在训练之前确定的,即它不是在训练期间学习的。 这种类型的激活函数在我们可能遭受稀疏梯度的任务中很受欢迎。

\begin{figure}[h]
  \centering
  \includegraphics[width = 0.8\textwidth]{CBL.png}
  \caption{CBL模块组成示意图}
  \label{cbl}
\end{figure}

\begin{figure}[h]
  \centering
  \includegraphics[width = 0.5\textwidth]{leakyrelu.png}
  \caption{Leaky ReLU示意图}
  \label{relu}
\end{figure}

\subsubsection{CSP模块}
SP结构从网络结构设计的角度来解决以往工作在推理过程中需要很大计算量的问题,CSP结构认为推理计算过高的问题是由于网络优化中的梯度信息重复导致的。CSP结构通过将基础层的特征图划分为两个部分,然后通过CSP结构将它们合并,可以在能够实现更丰富的梯度组合的同时减少计算量。
Yolov5使用CSPDarknet作为Backbone,从输入图像中提取丰富的信息特征。CSPNet解决了其他大型卷积神经网络框架Backbone中网络优化的梯度信息重复问题,将梯度的变化从头到尾地集成到特征图中,因此减少了模型的参数量和FLOPS数值,既保证了推理速度和准确率,又减小了模型尺寸。

\begin{figure}[h]
  \centering
  \includegraphics[width = 0.8\textwidth]{CSP.png}
  \caption{CSP模块组成示意图}
  \label{csp}
\end{figure}

\subsection{Neck}
Neck部分主要由FPN和PAN组成。FPN即特征金字塔网络(Feature Pyramid Network),特征金字塔是一种
用于检测不同尺度物体的系统。该算法模块利用固有的多尺度,
深度卷积网络的金字塔层次结构以边际额外成本构建特征金字塔。开发了具有横向连接的自上而下的架构,用于
在所有尺度上构建高级语义特征图。这种架构作为通用特征提取器在不同场景均显示出出色的性能。
FPN对小尺度目标检测效果更好。FPN可以利用经过top-down模型后的那些上下文信息(高层语义信息)。
对于小目标而言,FPN增加了特征映射的分辨率(即在更大的feature map上面进行操作,这样可以获得更多关于小目标的有用信息)。

PANet是一种旨在提升信息的路径聚合网络结构(Path Aggregation Network)。
具体来说,PANet增强了整个特征层次结构中
自下而上的低层精确定位信号,提出了自适应特征池,
它将特征网格和所有特征级别联系起来,进而在每个特征级别中生成有用的信息
直接传播到以下提议子网,缩短了较低层和最顶层特征之间的信息路径。
PANet在实例分割和目标检测等任务中均有着优秀的性能表现。

\subsection{Prediction}
Prediction部分包含上级输出的特征向量,并通过输入的特征向量得出坐标值、置信度,结合损失函数和后处理函数得出检测结果。

\section{YOLOv5基础网络性能验证与分析}
本节将在自建数据集上对YOLOv5基础网络的性能进行验证,将其与常见算法SSD和Faster RCNN进行对比,并分析YOLOv5算法的优势。

\subsection{训练环境与参数}
本章提到的YOLOv5基础算法用深度学习网络框架Pytorch进行实现、训练和测试。操作系统为Windows 10,使用的主要软件为python。实验平台采用的CPU型号为Intel Core i7-11700k,GPU型号为NVIDIA GeForce RTX 3080Ti,显存容量为12GB。训练过程中使用GPU进行加速。

\subsection{不同网络结构对比分析}
为了证明基础YOLOv5算法在红外无人机目标检测任务中的有效性,本节将其与目标检测算法SSD以及Faster RCNN进行对比实验。

\begin{table}[htbp]
  \caption{不同算法对红外无人机数据集的检测结果}
  \vspace{0.5em}\centering\wuhao
  \begin{tabular}{ccc}
  \toprule
  检测算法 & mAP & 推理时间\\
  \midrule
   5 & 269.8 & 0.000674\\
  10 & 421.0 & 0.001035\\
  20 & 640.2 & 0.001565\\
  \bottomrule
  \end{tabular}
\end{table}

\section{数据增强算法}
由于深度学习算法通常需要大量的数据进行学习,因此在原始数据集的基础上,将基础网络结合数据增强算法能在一定程度上提升算法的性能。因此本节将对常见的图像增强算法进行实验,在实验结果的基础上针对红外图像目标检测任务提出一种图像填充算法,此外还将针对红外无人机目标检测任务中的小目标占比较高的特性提出一种图像拼接的数据增强算法。

\subsection{主流的图像增强算法}
图像增强算法根据滤波方法不同分为空域增强算法和频域增强算法。空域增强算法是对图像的像素直接进行操作,常用的有直方图均衡、中值滤波器、拉普拉斯变化等,频域增强算法是以修改图像傅里叶变换为基础的,常用的有高通滤波器、低通滤波器等。由于空域增强算法相对简单,处理速度快,因此本课题采用的图像预处理算法都为空域增强算法,重点研究了几种具有代表性的空域增强算法。

由于空间域增强算法是采用不同的操作直接对像素进行处理,因此可以定义为:
\begin{equation}
  g(x, y)=T[f(x, y)]
\end{equation}

其中$f(x, y)$代表原始图像,$T$是作用在原图$(x, y)$邻域上的一种变换,$g(x, y)$表示处理过的图像。

\subsubsection{Inversion}
由于主流的目标检测算法应用的场景都是基于RGB图像的,不适于检测红外目标,因此需要将红外图像进行预处理以达到使红外图像更接近RGB图像的目的,通过域迁移的思想使得网络能够更加适应处理后的红外图像。一般用于目标检测所用的RGB图像都是白天所摄,通常情况是背景较亮,目标较暗。但是红外图像成像为辐射特性,故一般背景辐射较弱而目标辐射较强。因此,采用inversion操作:
\begin{equation}
  f: x_{p}=1-x
\end{equation}

\subsubsection{直方图均衡}
一般来说由于红外图像的对比度比较低因此导致其灰度分布通常都是分布在较窄的区域,与RGB图像的灰度分布不同,采用直方图均衡能够使红外图像的灰度分布更均匀从而达到增加图像细节信息的作用。针对直方图进行处理是多数空域处理技术中的重要手段,直方图均衡是对图像的灰度直方图进行非线性的拉伸,重新分配红外图像中的灰度值,从而增大红外图像的对比度,直方图均衡的步骤如下: 

(1)计算像素区间在$[0,L-1]$的图像灰度为:
\begin{equation}
  \mathrm{p}_{r}\left(r_{k}\right)=\frac{n_{k}}{n} \quad k=0,1,2, \ldots, L-1
\end{equation}

其中$n$代表图像的像素总数,$n_{k}$代表灰度级为$r_{k}$的像素个数。

(2)计算像素变换后的分布:
\begin{equation}
  \mathrm{s}_{k}=T\left(r_{k}\right)=\sum_{j=0}^{k} p_{r}\left(r_{j}\right)=\sum_{j=0}^{k} \frac{n_{j}}{n} \quad k=0,1,2, \ldots L-1
  \label{pix trans}
\end{equation}

通过式\ref{pix trans}可将图像中值为$r_{k}$的像素点映射到输出图像中值为$s_{k}$的点,此时$s_{k}$的像素值在$[0,1]$之间。

(2)将$s_{k}$转化为与原始图像灰度级一致:
\begin{equation}
  s_{k}=s_{k} *(L-1)
\end{equation}

直方图均衡对于太暗或太亮的图片都有很好的提高对比度效果,并且计算量不大,但该方法存在的局限在于变换后的图像灰度级会减少,造成细节信息的损失,严重的可能造成原本图像中的小目标丢失,后来针对以上问题发展出自适应直方图均衡(AHE),对比度受限的直方图均衡(CLAHE)等。

\subsubsection{USM锐化}
对于大多数情况下的目标检测任务,目标相对背景都是有较明显的轮廓,因此可以采用锐化处理方法进一步加强物体的边界,进而增强检测效果。

锐化的本质是对原图像实现高通滤波,得到原图像的高频分量,也就是对原图像中的边缘部分进行了增强。图像处理实现锐化有一种常用的算法USM(Unsharp Mask),这种锐化的方法就是对原图像先做一个高斯模糊取得原图像的低通分量,然后用原来的图像减去一个系数乘以高斯模糊之后的图像,然后再把值Scale到$[0,255]$的RGB像素值范围之内。基于USM锐化的方法可以去除一些细小的干扰细节和噪声,比一般直接使用卷积锐化算子得到的图像锐化结果更加可靠。
\begin{equation}
  I_{sharpend}=I_{original}+(I_{original}-I_{blurred})*a
\end{equation}

式中$I_{sharpend}$表示锐化后的图像,$I_{original}$表示原始图像,$I_{blurred}$表示模糊后的原始图像,$a$表示权重系数。

\subsection{通道填充算法}
如前文所述,红外图像由于信噪比低等特点因此相对可见光图像来说信息量
少。本章针对红外图像的特点,结合前文介绍的图像增强手段,提出了一种基于
通道填充的红外图像数据增强算法,该算法的主要思想为:由于红外图像为单通
道图像,携带的信息量有限,选择另外两种红外图像增强方法将单通道的红外图像填充成
三通道,再输入到网络中,增加数据量的同时使网络能够学到更丰富的特征。算
法的流程如图所示。

\begin{figure}[h]
  \centering
  \includegraphics[width = 0.8\textwidth]{通道填充流程图.png}
  \caption{通道填充算法流程图}
  \label{tdtc}
\end{figure}

首先将输入的图像分别进行两次图像增强处理,然后将增强后的图像与原图
进行融合,融合成三通道图片,由于不能准确知道哪些增强方法会使检测结果提
高,因此采取多种增强方法进行对比实验,包括以下方法:

(1)Inversion。主要目的在于网络初始化权重为在可见光图像进行预训练
后的权重,因此将红外图像进行 inversion 处理后,能够更接近可见光图像的灰
度图,从而达到提升检测效果的目的。

(2)CLAHE。与直方图均衡相似,只是处理的区域由原图变得更细,增强
对比度的同时,能够抑制噪声。CLAHE窗口大小选择 3*3。

(3)USM变换。主要目的在于抑制噪声的同时进行图像的锐化,强化目标的边缘。本文使用高斯模糊后的USM变换进行锐化。

\begin{figure}[!h]
  \setlength{\subfigcapskip}{-1bp}

  \centering
  \begin{minipage}{\textwidth}
  \centering
  \subfigure{\label{tdtc11}}\addtocounter{subfigure}{-2}
  \subfigure{\subfigure[正常目标示例~1]{\includegraphics[width=0.4\textwidth]{1.png}}}
  \hspace{2em}
  \subfigure{\label{tdtc12}}\addtocounter{subfigure}{-2}
  \subfigure{\subfigure[正常目标示例~2]{\includegraphics[width=0.4\textwidth]{2.png}}}
  \end{minipage}

  \centering
  \begin{minipage}{\textwidth}
  \centering
  \subfigure{\label{tdtc21}}\addtocounter{subfigure}{-2}
  \subfigure{\subfigure[小目标示例~1]{\includegraphics[width=0.4\textwidth]{5.png}}}
  \hspace{2em}
  \subfigure{\label{tdtc22}}\addtocounter{subfigure}{-2}
  \subfigure{\subfigure[小目标示例~2]{\includegraphics[width=0.4\textwidth]{6.png}}}
  \end{minipage}

  \centering
  \begin{minipage}{\textwidth}
  \centering
  \subfigure{\label{tdtc31}}\addtocounter{subfigure}{-2}
  \subfigure{\subfigure[遮挡目标示例~1]{\includegraphics[width=0.4\textwidth]{3.png}}}
  \hspace{2em}
  \subfigure{\label{tdtc32}}\addtocounter{subfigure}{-2}
  \subfigure{\subfigure[遮挡目标示例~2]{\includegraphics[width=0.4\textwidth]{4.png}}}
  \end{minipage}

  \centering
  \begin{minipage}{\textwidth}
  \centering
  \subfigure{\label{tdtc41}}\addtocounter{subfigure}{-2}
  \subfigure{\subfigure[遮挡目标示例~1]{\includegraphics[width=0.4\textwidth]{3.png}}}
  \hspace{2em}
  \subfigure{\label{tdtc42}}\addtocounter{subfigure}{-2}
  \subfigure{\subfigure[遮挡目标示例~2]{\includegraphics[width=0.4\textwidth]{4.png}}}
  \end{minipage}

  \vspace{0.2em}
  \caption{数据集图像示例}
  {\label{tdtc2}}
\end{figure}

增强后的图片效果如图\ref{tdtc2}所示,其中\ref{tdtc11}和\ref{tdtc12}表示原图像,\ref{tdtc21}和\ref{tdtc22}表示原图经过Inversion处理后的图像,\ref{tdtc31}和\ref{tdtc32}表示原图像经过CLAHE处理后的图像,\ref{tdtc41}和\ref{tdtc42}表示原图像经过USM处理后的图像。

\subsection{图像拼接算法}
由于在红外无人机目标检测的场景下,小尺度目标(面积占整个图像总面积比例较小)的数量占比较大,因此考虑采用一种图像拼接的数据增强算法,将训练数据中的小目标增多,从而增强网络对小目标的检测能力。

该算法的实现方法是,从训练集中随机抽取4个图像,将这4张图像进行拼接后生成一张与原始图像相同大小的新图像输入网络。

\begin{figure}[h]
  \centering
  \includegraphics[width = 0.8\textwidth]{通道填充流程图.png}
  \caption{图像拼接示意图}
  \label{txpj}
\end{figure}

\section{改进损失函数}
YOLOv5的损失函数由三部分组成:第一部分是bounding box损失,主要根据预测目标框和真值目标框的重合程度进行计算。第二部分是置信度损失,第三部分是分类误差。

其中对于bounding box的损失函数主要是基于IoU(交并比,Intersection over Union)进行计算。IoU一般表示预测框与真值框面积的交并比,即二者相交的面积除以二者所占的总面积。
\begin{equation}
  I o U=\frac{|A \cap B|}{|A \cup B|}
\end{equation}

这样一种计算方式相对于之前的基于点到点之间距离的MSE损失函数已经有较大的优势,主要体现在:

(1)MSE损失函数由于是按点距离计算,因此对于任务目标的尺寸变化较为敏感,尺寸变化往往会引起损失函数的无意义波动。

(2)MSE损失函数从点距离出发,忽略了目标框各点之间的关联。因此基于IoU的损失函数已经在一定程度上反映了预测框与真值框之间的误差关系。

而IoU损失函数有两个重要特性,使得它能作为很多计算机视觉算法的损失函数。

(1)$\mathcal{L}=1-IoU$满足非负性、同一性、对称性、三角不等性等所有损失函数的特性。

(2)IoU相对于问题的规模是不变的。这意味着两个任意形状之间的相似性独立于它们的空间规模。

然而基于IoU的损失函数也存在一定的问题。

(1)当预测框与真值框没有相交区域时,IoU值为0,此时损失loss为0,无法进行梯度回传,也就是无法进行模型训练。

(2)IoU还是无法精确地衡量预测框与真值框之间的误差情况,尤其是在IoU为0时,预测框与真实框之间的距离变得无法量化。

因此,针对IoU的这一问题,为了增强算法对红外无人机目标检测任务中的目标框定位能力,可以引入一种更加完善的GIoU损失函数。设$I o U=\frac{\mathcal{I}}{\mathcal{U}}$
,其中$\mathcal{I}$表示预测框与真值框相交部分面积,$\mathcal{U}$表示预测框与真值框合并部分面积,则GIoU的计算公式是:
\begin{equation}
  G I o U=I o U-\frac{A^{c}-\mathcal{U}}{A^{c}}
\end{equation}

其中$A^{c}$表示预测框和真值框的最小闭包区域的面积。

GIoU的计算流程是:

(1)




\sindex[china]{du!段誉}引文标注遵照GB/T7714-2005,采用顺序编码制。正文中引用文献的标示应置于所引内容最后一个字的右上角,所引文献编号用阿拉伯数字置于方括号“[ ]”中,用小4号字体的上角标。要求:

(1)引用单篇文献时,如“二次铣削\cite{cnproceed}”。

(2)同一处引用多篇文献时,各篇文献的序号在方括号内全部列出,各序号间用“,”,如
遇连续序号,可标注讫序号。如,…形成了多种数学模型\cite{cnarticle,cnproceed}…
注意此处添加\cs{inlinecite}中文空格\inlinecite{cnarticle,cnproceed},可以在cfg文件中修改空格类型。

(3)多次引用同一文献时,在文献序号的“[ ]”后标注引文页码。如,…间质细胞CAMP含量
测定\cite[100-197]{cnarticle}…。…含量测定方法规定
\cite[92]{cnarticle}…。

(4)当提及的参考文献为文中直接说明时,则用小4号字与正文排齐,如“由文献\inlinecite{hithesis2017}可知”

\section{定理和定义等}[Theorem]
\begin{theorem}[\cite{cnproceed}]
宇宙大爆炸是一种爆炸。
\end{theorem}
\begin{definition}[(霍金)]
宇宙大爆炸是一种爆炸。
\end{definition}
\begin{assumption}
宇宙大爆炸是一种爆炸。
\end{assumption}
\begin{lemma}
宇宙大爆炸是一种爆炸。
\end{lemma}
\begin{corollary}
宇宙大爆炸是一种爆炸。
\end{corollary}
\begin{exercise}
宇宙大爆炸是一种爆炸。
\end{exercise}
\begin{problem}[(Albert Einstein)]
宇宙大爆炸是一种爆炸。
\end{problem}
\begin{remark}
宇宙大爆炸是一种爆炸。
\end{remark}
\begin{axiom}[(爱因斯坦)]
宇宙大爆炸是一种爆炸。
\end{axiom}
\begin{conjecture}
宇宙大爆炸是一种爆炸。
\end{conjecture}
\section{图片}[Pictures]
图应有自明性。插图应与文字紧密配合,文图相符,内容正确。选图要力求精练,插图、照
片应完整清晰。机械工程图:采用第一角投影法,严格按照GB4457~GB131-83《机械制图》
标准规定。数据流程图、程序流程图、系统流程图等按GB1526-89标准规定。电气图:图形
符号、文字符号等应符合附录3所列有关标准的规定。流程图:必须采用结构化程序并正确
运用流程框图。对无规定符号的图形应采用该行业的常用画法。坐标图的坐标线均用细实线
,粗细不得超过图中曲线;有数字标注的坐标图,必须注明坐标单位。照片图要求主题和主
要显示部分的轮廓鲜明,便于制版。如用放大或缩小的复制品,必须清晰,反差适中。照片
上应有表示目的物尺寸的标度。引用文献中的图时,除在正文文字中标注参考文献序号以外
,还必须在中、英文表题的右上角标注参考文献序号。

\subsection{博士毕业论文双语题注}[Doctoral picture example]
\begin{figure}[htpb]
\centering
\includegraphics[width = 0.4\textwidth]{golfer}
\bicaption[golfer1]{}{打高尔夫球球的人(博士论文双语题注)}{Fig.$\!$}{The person playing golf (Doctoral thesis)}
\end{figure}

每个图均应有图题(由图序和图名组成),图题不宜有标点符号,图名在图序之后空1个半
角字符排写。图序按章编排,如第1章第一个插图的图号为“图1-1”。图题置于图下,硕士论
文只用中文,博士论文用中、英两种文字,居中书写,中文在上,要求中文用宋体5号字,
英文用Times New Roman 5号字。有图注或其它说明时应置于图题之上。引用图应注明出处
,在图题右上角加引用文献号。图中若有分图时,分图题置于分图之下或图题之下,可以只
用中文书写,分图号用a)、b)等表示。图中各部分说明应采用中文(引用的外文图除外)或
数字符号,各项文字说明置于图题之上(有分图时,置于分图题之上)。图中文字用宋体、
Times New Roman字体,字号尽量采用5号字(当字数较多时可用小5号字,以清晰表达为原
则,但在一个插图内字号要统一)。同一图内使用文字应统一。图表中物理量、符号用斜体
。
\subsection{本硕论文题注}[Other picture example]
\begin{figure}[h]
\centering
\includegraphics[width = 0.4\textwidth]{golfer}
\caption{打高尔夫球的人,硕士论文要求只用汉语}
\end{figure}

\subsection{并排图和子图}[Abreast-picture and Sub-picture example]
\subsubsection{并排图}[Abreast-picture example]

使用并排图时,需要注意对齐方式。默认情况是中部对齐。这里给出中部对齐、顶部对齐
、图片底部对齐三种常见方式。其中,底部对齐方式有一个很巧妙的方式,将长度比较小
的图放在左面即可。

\begin{figure}[htbp]
\centering
\begin{minipage}{0.4\textwidth}
\centering
\includegraphics[width=\textwidth]{golfer}
\bicaption[golfer2]{}{打高尔夫球的人}{Fig.$\!$}{The person playing golf}
\end{minipage}
\centering
\begin{minipage}{0.4\textwidth}
\centering
\includegraphics[width=\textwidth]{golfer}
\bicaption[golfer3]{}{打高尔夫球的人。注意,这里默认居中}{Fig.$\!$}{The person playing golf. Please note that, it is vertically center aligned by default.}
\end{minipage}
\end{figure}

\begin{figure}[htbp]
\centering
\begin{minipage}[t]{0.4\textwidth}
\centering
\includegraphics[width=\textwidth]{golfer}
\bicaption[golfer5]{}{打高尔夫球的人}{Fig.$\!$}{The person playing golf}
\end{minipage}
\centering
\begin{minipage}[t]{0.4\textwidth}
\centering
\includegraphics[width=\textwidth]{golfer}
\bicaption[golfer8]{}{打高尔夫球的人。注意,此图是顶部对齐}{Fig.$\!$}{The person playing golf. Please note that, it is vertically top aligned.}
\end{minipage}
\end{figure}

\begin{figure}[htbp]
\centering
\begin{minipage}[t]{0.4\textwidth}
\centering
\includegraphics[width=\textwidth,height=\textwidth]{golfer}
\bicaption[golfer9]{}{打高尔夫球的人。注意,此图对齐方式是图片底部对齐}{Fig.$\!$}{The person playing golf. Please note that, it is vertically bottom aligned for figure.}
\end{minipage}
\centering
\begin{minipage}[t]{0.4\textwidth}
\centering
\includegraphics[width=\textwidth]{golfer}
\bicaption[golfer6]{}{打高尔夫球的人}{Fig.$\!$}{The person playing golf}
\end{minipage}
\end{figure}

\subsubsection{子图}[Sub-picture example]
注意:子图题注也可以只用中文。规范规定“分图题置于分图之下或图题之下”,但没有给出具体的格式要求。
没有要求的另外一个说法就是“无论什么格式都不对”。
所以只有在一个图中有标注“a),b)”,无法使用\cs{subfigure}的情况下,使用最后一个图例中的格式设置方法,否则不要使用。
为了应对“无论什么格式都不对”,这个子图图题使用“minipage”和“description”环境,宽度,对齐方式可以按照个人喜好自由设置,是否使用双语子图图题也可以自由设置。

\begin{figure}[!h]
\setlength{\subfigcapskip}{-1bp}
\centering
\begin{minipage}{\textwidth}
\centering
\subfigure{\label{golfer41}}\addtocounter{subfigure}{-2}
\subfigure[The person playing golf]{\subfigure[打高尔夫球的人~1]{\includegraphics[width=0.4\textwidth]{golfer}}}
\hspace{2em}
\subfigure{\label{golfer42}}\addtocounter{subfigure}{-2}
\subfigure[The person playing golf]{\subfigure[打高尔夫球的人~2]{\includegraphics[width=0.4\textwidth]{golfer}}}
\end{minipage}
\centering
\begin{minipage}{\textwidth}
\centering
\subfigure{\label{golfer43}}\addtocounter{subfigure}{-2}
\subfigure[The person playing golf]{\subfigure[打高尔夫球的人~3]{\includegraphics[width=0.4\textwidth]{golfer}}}
\hspace{2em}
\subfigure{\label{golfer44}}\addtocounter{subfigure}{-2}
\subfigure[The person playing golf. Here, 'hang indent' and 'center last line' are not stipulated in the regulation.]{\subfigure[打高尔夫球的人~4。注意,规范中没有明确规定要悬挂缩进、最后一行居中。]{\includegraphics[width=0.4\textwidth]{golfer}}}
\end{minipage}
\vspace{0.2em}
\bicaption[golfer4]{}{打高尔夫球的人}{Fig.$\!$}{The person playing gol}
\end{figure}

\begin{figure}[t]
  \centering
  \begin{minipage}{.7\linewidth}
    \setlength{\subfigcapskip}{-1bp}
    \centering
    \begin{minipage}{\textwidth}
      \centering
      \subfigure{\label{golfer45}}\addtocounter{subfigure}{-2}
      \subfigure[The person playing golf]{\subfigure[打高尔夫球的人~1]{\includegraphics[width=0.4\textwidth]{golfer}}}
      \hspace{4em}
      \subfigure{\label{golfer46}}\addtocounter{subfigure}{-2}
      \subfigure[The person playing golf]{\subfigure[打高尔夫球的人~2]{\includegraphics[width=0.4\textwidth]{golfer}}}
    \end{minipage}
    \vskip 0.2em
  \wuhao 注意:这里是中文图注添加位置(我工要求,图注在图题之上)。
    \vspace{0.2em}
\bicaption[golfer47]{}{打高尔夫球的人。注意,此处我工有另外一处要求,子图图题可以位于主图题之下。但由于没有明确说明位于下方具体是什么格式,所以这里不给出举例。}{Fig.$\!$}{The person playing golf. Please note that, although it is appropriate to put subfigures' captions under this caption as stipulated in regulation, but its format is not clearly stated.}
  \end{minipage}
\end{figure}

\begin{figure}[t]
\centering
\begin{tikzpicture}
	\node[anchor=south west,inner sep=0] (image) at (0,0) {\includegraphics[width=0.3\textwidth]{golfer}};
	\begin{scope}[x={(image.south east)},y={(image.north west)}]
		\node at (0.3,0.5) {a)};
		\node at (0.8,0.2) {b)};
	\end{scope}
\end{tikzpicture}
\bicaption[golfer0]{}{打高尔夫球球的人(博士论文双语题注)}{Fig.$\!$}{The person playing golf (Doctoral thesis)}
\vskip -0.4em
 \hspace{2em}
\begin{minipage}[t]{0.3\textwidth}
\wuhao \setlist[description]{font=\normalfont}
	\begin{description}
		\item[(a)]子图图题
	\end{description}
 \end{minipage}
 \hspace{2em}
 \begin{minipage}[t]{0.3\textwidth}
\wuhao \setlist[description]{font=\normalfont}
	\begin{description}
		\item[(b)]子图图题
		\item[(b)]Subfigure caption
	\end{description}
\end{minipage}
\end{figure}


\begin{figure}[!h]
	\centering
	\begin{sideways}
		\begin{minipage}{\textheight}
			\centering
			\fbox{\includegraphics[width=0.2\textwidth]{golfer}}
			\fbox{\includegraphics[width=0.2\textwidth]{golfer}}
			\fbox{\includegraphics[width=0.2\textwidth]{golfer}}
			\fbox{\includegraphics[width=0.2\textwidth]{golfer}}
			\fbox{\includegraphics[width=0.2\textwidth]{golfer}}
			\fbox{\includegraphics[width=0.2\textwidth]{golfer}}
			\fbox{\includegraphics[width=0.2\textwidth]{golfer}}
\bicaption[golfer7]{}{打高尔夫球的人(非规范要求)}{Fig.$\!$}{The person playing golf (Not stated in the regulation)}
		\end{minipage}
	\end{sideways}
\end{figure}

\clearpage

如果不想让图片浮动到下一章节,那么在此处使用\cs{clearpage}命令。

\section{如何做出符合规范的漂亮的图}
关于作图工具在后文\ref{drawtool}中给出一些作图工具的介绍,此处不多言。
此处以R语言和Tikz为例说明如何做出符合规范的图。

\subsection{Tikz作图举例}
使用Tikz作图核心思想是把格式、主题、样式与内容分离,定义在全局中。
注意字体设置可以有两种选择,如何字少,用五号字,字多用小五。
使用Tikz作图不会出现字体问题,字体会自动与正文一致。

\begin{figure}[thb!]
  \centering
      \begin{tikzpicture}[xscale=0.8,yscale=0.3,rotate=90]
        \small
	\draw (-22,6.5) node[refcell]{参考基因组};
	\draw[refline] (-23, 5) -- (27, 5);
	\draw (-22,3.75) node[tscell]{肿瘤样本};
	\draw (-20,3.75) node[tncell]{正常细胞};
	\draw[tnline] (-21, 2.5) -- (27, 2.5);
	\draw (-20,1.25) node[ttcell]{肿瘤细胞};
	\rcell{2}{6};
	\draw[fakeevolve] (4.5, 5.25) -- (4.5, 4.8);
	\ncell{2}{4};
	\draw[evolve] (4.5, 3) .. controls (4.5,2.8) and (-3.5,2.9) ..  (-3.5, 2);
	\draw[evolve] (4.5, 3) .. controls (4.5,2.8) and (11.5,2.9) .. (11.5, 2);
	\tcellone{-6}{1.5};
	\draw (-9, 2) node[ttcell]{1};
	\draw[evolve] (-3.5, 0) .. controls (-3.5,-0.2) and (-12,-0.1) .. (-12, -1.5);
	\draw[evolve] (-3.5, 0) .. controls (-3.5,-0.2) and (1.5,-0.1) .. (1.5, -1.5);
	\tcellthree{7}{1.5};
	\draw (4, 2) node[ttcell]{2};
	\draw[evolve] (11, 0.5) .. controls (11,0.3) and (19,0.4) .. (19, -1.5);
	\tcellfive{-16}{-2};
	\draw (-19, -1.5) node[ttcell]{3};
	\tcelltwo{-1}{-2};
	\draw (-4, -1.5) node[ttcell]{4};
	\tcellfour{12}{-2};
	\draw (9, -1.5) node[ttcell]{5};
      \end{tikzpicture}
  \begin{minipage}{.9\linewidth}
      \vskip 0.2em
      \wuhao 图中,带有箭头的淡蓝色箭头表示肿瘤子种群的进化方向。一般地,从肿瘤组织中取用于进行二代测序的样本中含有一定程度的正常细胞污染,因此肿瘤的样本中含有正常细胞和肿瘤细胞。每一个子种群的基因组的模拟过程是把生殖细胞变异和体细胞变异加入到参考基因组中。
      \vspace{0.6em}
  \end{minipage}
\bicaption[tumor]{}{肿瘤组织中各个子种群的进化示意图}{Fig.$\!$}{The diagram of tumor subpopulation evolution process}
\end{figure}

\subsection{R作图}
R是一种极具有代表性的典型的作图工具,应用广泛。
与Tikz图~\ref{tumor}~不同,R作图分两种情况:(1)可以转换为Tikz码;(2)不可转换为Tikz码。
第一种情况图形简单,图形中不含有很多数据点,使用R语言中的Tikz包即可。
第二种情况是图形复杂,含有海量数据点,这时候不要转成Tikz矢量图,这会使得论文体积巨大。
推荐使用pdf或png非矢量图形。
使用非矢量图形时要注意选择好字号(五号或小五),和字体(宋体、新罗马)然后选择生成图形大小,注意此时在正文中使用\cs{includegraphics}命令导入时,不要像导入矢量图那样控制图形大小,使用图形的原本的
宽度和高度,这样就确保了非矢量图形中的文字与正文一致了。

为了控制\hithesis\ 的大小,此处不给出具体举例,

\section{表格}

表应有自明性。表格不加左、右边线。表的编排建议采用国际通行的三线表。表中文字用宋
体~5~号字。每个表格均应有表题(由表序和表名组成)。表序一般按章编排,如第~1~章第
一个插表的序号为“表~1-1”等。表序与表名之间空一格,表名中不允许使用标点符号,表名
后不加标点。表题置于表上,硕士学位论文只用中文,博士学位论文用中、英文两种文字居
中排写,中文在上,要求中文用宋体~5~号字,英文用新罗马字体~5~号字。表头设计应简单
明了,尽量不用斜线。表头中可采用化学符号或物理量符号。


\subsection{普通表格的绘制方法}[Methods of drawing normal tables]

表格应具有三线表格式,因此需要调用~booktabs~宏包,其标准格式如表~\ref{table1}~所示。
\begin{table}[htbp]
\bicaption[table1]{}{符合研究生院绘图规范的表格}{Table$\!$}{Table in agreement of the standard from graduate school}
\vspace{0.5em}\centering\wuhao
\begin{tabular}{ccccc}
\toprule
$D$(in) & $P_u$(lbs) & $u_u$(in) & $\beta$ & $G_f$(psi.in)\\
\midrule
 5 & 269.8 & 0.000674 & 1.79 & 0.04089\\
10 & 421.0 & 0.001035 & 3.59 & 0.04089\\
20 & 640.2 & 0.001565 & 7.18 & 0.04089\\
\bottomrule
\end{tabular}
\end{table}
全表如用同一单位,则将单位符号移至表头右上角,加圆括号。表中数据应准确无误,书写
清楚。数字空缺的格内加横线“-”(占~2~个数字宽度)。表内文字或数字上、下或左、右
相同时,采用通栏处理方式,不允许用“〃”、“同上”之类的写法。表内文字说明,起行空一
格、转行顶格、句末不加标点。如某个表需要转页接排,在随后的各页上应重复表的编号。
编号后加“(续表)”,表题可省略。续表应重复表头。

\subsection{长表格的绘制方法}[Methods of drawing long tables]

长表格是当表格在当前页排不下而需要转页接排的情况下所采用的一种表格环境。若长表格
仍按照普通表格的绘制方法来获得,其所使用的\verb|table|浮动环境无法实现表格的换页
接排功能,表格下方过长部分会排在表格第1页的页脚以下。为了能够实现长表格的转页接
排功能,需要调用~longtable~宏包,由于长表格是跨页的文本内容,因此只需要单独的
\verb|longtable|环境,所绘制的长表格的格式如表~\ref{table2}~所示。

注意,长表格双语标题的格式。

\vspace{-1.5bp}
\ltfontsize{\wuhao[1.667]}
\wuhao[1.667]\begin{longtable}{ccc}%
\longbionenumcaption{}{{\wuhao 中国省级行政单位一览}\label{table2}}{Table$\!$}{}{{\wuhao Overview of the provincial administrative unit of China}}{-0.5em}{3.15bp}\\
%\caption{\wuhao 中国省级行政单位一览}\label{table2}\\
\toprule 名称 & 简称 & 省会或首府\\ \midrule
\endfirsthead
\multicolumn{3}{r}{表~\thetable(续表)}\vspace{0.5em}\\
\toprule 名称 & 简称 & 省会或首府\\ \midrule
\endhead
\midrule[0.5pt]
\endfoot
\bottomrule
\endlastfoot
北京市 & 京 & 北京\\
天津市 & 津 & 天津\\
河北省 & 冀 & 石家庄市\\
山西省 & 晋 & 太原市\\
内蒙古自治区 & 蒙 & 呼和浩特市\\
辽宁省 & 辽 & 沈阳市\\
吉林省 & 吉 & 长春市\\
黑龙江省 & 黑 & 哈尔滨市\\
上海市 & 沪/申 & 上海\\
江苏省 & 苏 & 南京市\\
浙江省 & 浙 & 杭州市\\
安徽省 & 皖 & 合肥市\\
福建省 & 闽 & 福州市\\
江西省 & 赣 & 南昌市\\
山东省 & 鲁 & 济南市\\
河南省 & 豫 & 郑州市\\
湖北省 & 鄂 & 武汉市\\
湖南省 & 湘 & 长沙市\\
广东省 & 粤 & 广州市\\
广西壮族自治区 & 桂 & 南宁市\\
海南省 & 琼 & 海口市\\
重庆市 & 渝 & 重庆\\
四川省 & 川/蜀 & 成都市\\
贵州省 & 黔/贵 & 贵阳市\\
云南省 & 云/滇 & 昆明市\\
西藏自治区 & 藏 & 拉萨市\\
陕西省 & 陕/秦 & 西安市\\
甘肃省 & 甘/陇 & 兰州市\\
青海省 & 青 & 西宁市\\
宁夏回族自治区 & 宁 & 银川市\\
新疆维吾尔自治区 & 新 & 乌鲁木齐市\\
香港特别行政区 & 港 & 香港\\
澳门特别行政区 & 澳 & 澳门\\
台湾省 & 台 & 台北市\\
\end{longtable}\normalsize
\vspace{-1em}

此长表格~\ref{table2}~第~2~页的标题“编号(续表)”和表头是通过代码自动添加上去的,无需人工添加,若表格在页面中的竖直位置发生了变化,长表格在第~2~页
及之后各页的标题和表头位置能够始终处于各页的最顶部,也无需人工调整,\LaTeX~系统的这一优点是~word~等软件所无法比拟的。

\subsection{列宽可调表格的绘制方法}[Methods of drawing tables with adjustable-width columns]
论文中能用到列宽可调表格的情况共有两种,一种是当插入的表格某一单元格内容过长以至
于一行放不下的情况,另一种是当对公式中首次出现的物理量符号进行注释的情况,这两种
情况都需要调用~tabularx~宏包。下面将分别对这两种情况下可调表格的绘制方法进行阐述
。
\subsubsection{表格内某单元格内容过长的情况}[The condition when the contents in
some cells of tables are too long]
首先给出这种情况下的一个例子如表~\ref{table3}~所示。
\begin{table}[htbp]
  \centering
\bicaption[table3]{}{最小的三个正整数的英文表示法}{Table$\!$}{The English construction of the smallest three positive integral numbers}\vspace{0.5em}\wuhao
\begin{tabularx}{0.7\textwidth}{llX}
\toprule
Value & Name & Alternate names, and names for sets of the given size\\
\midrule
1 & One & ace, single, singleton, unary, unit, unity\\
2 & Two & binary, brace, couple, couplet, distich, deuce, double, doubleton, duad, duality, duet, duo, dyad, pair, snake eyes, span, twain, twosome, yoke\\
3 & Three & deuce-ace, leash, set, tercet, ternary, ternion, terzetto, threesome, tierce, trey, triad, trine, trinity, trio, triplet, troika, hat-trick\\
\bottomrule
\end{tabularx}
\end{table}
tabularx环境共有两个必选参数:第1个参数用来确定表格的总宽度,第2个参数用来确定每
列格式,其中标为X的项表示该列的宽度可调,其宽度值由表格总宽度确定。标为X的列一般
选为单元格内容过长而无法置于一行的列,这样使得该列内容能够根据表格总宽度自动分行
。若列格式中存在不止一个X项,则这些标为X的列的列宽相同,因此,一般不将内容较短的
列设为X。标为X的列均为左对齐,因此其余列一般选为l(左对齐),这样可使得表格美观
,但也可以选为c或r。

\subsubsection{对物理量符号进行注释的情况}[The condition when physical symbols
need to be annotated]

为使得对公式中物理量符号注释的转行与破折号“———”后第一个字对齐,此处最好采用表格
环境。此表格无任何线条,左对齐,且在破折号处对齐,一共有“式中”二字、物理量符号和
注释三列,表格的总宽度可选为文本宽度,因此应该采用\verb|tabularx|环境。由
\verb|tabularx|环境生成的对公式中物理量符号进行注释的公式如式(\ref{eq:1})所示。
\begin{equation}\label{eq:1}
\ddot{\boldsymbol{\rho}}-\frac{\mu}{R_{t}^{3}}\left(3\mathbf{R_{t}}\frac{\mathbf{R_{t}\rho}}{R_{t}^{2}}-\boldsymbol{\rho}\right)=\mathbf{a}
\end{equation}
\begin{tabularx}{\textwidth}{@{}l@{\quad}r@{———}X@{}}
式中& $\boldsymbol{\rho}$ &追踪飞行器与目标飞行器之间的相对位置矢量;\\
&  $\boldsymbol{\ddot{\rho}}$&追踪飞行器与目标飞行器之间的相对加速度;\\
&  $\mathbf{a}$   &推力所产生的加速度;\\
&  $\mathbf{R_t}$ & 目标飞行器在惯性坐标系中的位置矢量;\\
&  $\omega_{t}$ & 目标飞行器的轨道角速度;\\
&  $\mathbf{g}$ & 重力加速度,$=\frac{\mu}{R_{t}^{3}}\left(
3\mathbf{R_{t}}\frac{\mathbf{R_{t}\rho}}{R_{t}^{2}}-\boldsymbol{\rho}\right)=\omega_{t}^{2}\frac{R_{t}}{p}\left(
3\mathbf{R_{t}}\frac{\mathbf{R_{t}\rho}}{R_{t}^{2}}-\boldsymbol{\rho}\right)$,这里~$p$~是目标飞行器的轨道半通径。
\end{tabularx}\vspace{3.15bp}
由此方法生成的注释内容应紧邻待注释公式并置于其下方,因此不能将代码放入
\verb|table|浮动环境中。但此方法不能实现自动转页接排,可能会在当前页剩余空间不够
时,全部移动到下一页而导致当前页出现很大空白。因此在需要转页处理时,还请您手动将
需要转页的代码放入一个新的\verb|tabularx|环境中,将原来的一个\verb|tabularx|环境
拆分为两个\verb|tabularx|环境。

\subsubsection{排版横版表格的举例}[An example of landscape table]

\begin{table}[p]
\centering
\begin{sideways}
\begin{minipage}{\textheight}
\bicaption[table4]{}{不在规范中规定的横版表格}{Table$\!$}{A table style which is not stated in the regulation}
\vspace{0.5em}\centering\wuhao
\begin{tabular}{ccccc}
\toprule
$D$(in) & $P_u$(lbs) & $u_u$(in) & $\beta$ & $G_f$(psi.in)\\
\midrule
 5 & 269.8 & 0.000674 & 1.79 & 0.04089\\
10 & 421.0 & 0.001035 & 3.59 & 0.04089\\
20 & 640.2 & 0.001565 & 7.18 & 0.04089\\
\bottomrule
\end{tabular}
\end{minipage}
\end{sideways}
\end{table}


\section{公式}
与正常\LaTeX\ 使用方法一致,此处略。关于公式中符号样式的定义在`hithesis.sty'有示
例。

\section{其他杂项}[Miscellaneous]

\subsection{右翻页}[Open right]

对于双面打印的论文,强制使每章的标题页出现右手边为右翻页。
规范中没有明确规定是否是右翻页打印。
模板给出了右翻页选项。
为了应对用户的个人喜好,在希望设置成右翻页的位置之前添加\cs{cleardoublepage}命令即可。

\subsection{算法}[Algorithms]
我工算法有以下几大特点。

(1)算法不在规范中要求。

(2)算法常常被使用(至少计算机学院)。

(3)格式乱,甚至出现了每个实验室的格式要求都不一样。

此处不给出示例,因为没法给,在
\href{https://github.com/dustincys/PlutoThesis}{https://github.com/dustincys/PlutoThesis}
的readme文件中有不同实验室算法要求说明。

\subsection{脚注}[Footnotes]
不在再规范\footnote{规范是指\PGR\ 和\UGR}中要求,模板默认使用清华大学的格式。

\subsection{源码}[Source code]
也不在再规范中要求。如果有需要最好使用minted包,但在编译的时候需要添加“
-shell-escape”选项且安装pygmentize软件,这些不在模板中默认载入,如果需要自行载入
。
\subsection{思源宋体}[Siyuan font]
如果要使用思源字体,需要思源字体的定义文件,此文件请到模板的开发版网址github:
\href{https://github.com/dustincys/hithesis}{https://github.com/dustincys/hithesis}
或者oschia:
\href{https://git.oschina.net/dustincys/hithesis}{https://git.oschina.net/dustincys/hithesis}
处下载。

\subsection{专业绘图工具}[Processional drawing tool]
\label{drawtool}
推荐使用tikz包,使用tikz源码绘图的好处是,图片中的字体与正文中的字体一致。具体如
何使用tikz绘图不属于模板范畴。
tikz适合用来画不需要大量实验数据支撑示意图。但R语言等专业绘图工具具有画出各种、
专业、复杂的数据图。R语言中有tikz包,能自动生成tikz码,这样tikz几乎无所不能。
对于排版有极致追求的小伙伴,可以参考
\href{http://www.texample.net/tikz/resources/}{http://www.texample.net/tikz/resources/}
中所列工具,几乎所有作图软件所作的图形都可转成tikz,然后可以自由的在tikz中修改
图中内容,定义字体等等。实现前文窝工规范中要求的图中字体的一致性的终极目标。


\subsection{术语词汇管理}[Manage glossaries]
推荐使用glossaries包管理术语、缩略语,可以自动生成首次全写,非首次缩写。

\subsection{\TeX\ 源码编辑器}[\TeX editor]
推荐:(1)付费软件Winedt;(2)免费软件kile;(3)vim或emaces或sublime等神级编
译器(需要配置)。

\subsection{\LaTeX\ 排版重要原则}[\LaTeX\ typesetting rules]
格式和内容分离是\LaTeX\ 最大优势,所有多次出现的内容、样式等等都可以定义为简单命
令、环境。这样的好处是方便修改、管理。例如,如果想要把所有的表示向量的符号由粗体
\cs{mathbf}变换到花体\cs{mathcal},只需修改该格式的命令的定义部分,不需要像MS
word那样处处修改。总而言之,使用自定义命令和环境才是正确的使用\LaTeX\ 的方式。

\section{关于捐助}
各位刀客和大侠如用的嗨,要解囊相助,请参照图~\ref{zfb}~中提示操作(二维码被矢量化后之后去
除了头像等冗余无用的部分~)。

\begin{figure}[!h]
\centering\includegraphics[width=0.4\textwidth]{zfb}
\vspace{0.2em}
\bicaption[Donation]{}{捐助,注意此处是子图只用汉语图题的形式,我工规定可以不用
英语图题}{Fig.$\!$}{Donation, please note that it is OK to use Chinese caption
only}
\end{figure}


% Local Variables:
% TeX-master: "../main"
% TeX-engine: xetex
% End:
