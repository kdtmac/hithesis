% !Mode:: "TeX:UTF-8" 
\begin{conclusions}

\section{论文工作总结}
本文提出了一种基于深度学习的算法来完成红外图像无人机目标检测任务。对比目前已有的目标检测算法,本文提出的检测算法,在红外无人机目标数据集上能取得更高的检测精度并且具备更快的推理速度,本文提出的检测算法可以在嵌入式设备上运行,并且检测速度基本达到了实时推理的要求。

本文所做的工作和取得的主要结果如下:

(1)由于目前开源的红外无人机数据集匮乏,因此本文标注并制作了一套红外无人机数据集。

(2)由于红外图像普遍都存在目标边缘模糊、分辨率低等特点,目前主流的目标检测
算法在红外图像上的检测精度不高,因此在红外
图像的输入到网络进行训练之前需要对红外图像进行数据增强。本文研究了常用的图像增强算法,提出了一种基于通道填充的红外图像数据增强算法,该算
法将原始图像和两个增强过的图像合成BGR三通道输出最终的增强图像,并作为增强数据输入到网络中进行训练。在本文自建红外无人机数据集上的测试结果表明这一数据增强方法提升了算法的检测精度。

(3)由于红外图像无人机目标检测中小目标的出现频率较高,因此一般的算法会因为对小目标检测能力较差而无法胜任红外无人机检测任务。本文针对这一问题设计了基于图像拼接的数据增强算法,将4张原始图像进行拼接后合成新图像输入网络进行训练,增强网络对小目标的检测能力和对复杂背景的辨别能力。在本文自建红外无人机数据集上的测试结果表明这一数据增强方法在结合网络结构的改进后,可以提升算法的检测精度。

(4)针对红外无人机目标检测任务中,小目标出现频率较高,使得目标检测算法容易存在定位不准的问题,从YOLOv5模型的损失函数着手进行改进,设计 GIoU 损失函数代替原来的
IoU损失函数。在本文自建红外无人机数据集上的测试结果表明这一改进可以提升算法的检测精度


(5)本文将研究得到的红外无人机检测算法进行了轻量化,并且在嵌入式设备上进行了算法功能的验证。本文针对嵌入式设备算力有限、检测任务对时间较敏感等问题,基于ghost网络模块对本文提出的红外无人机目标检测算法进行了轻量化,并且将算法的网络模型进行了转换,最终在NVIDIA AGX XAVIER平台上进行了算法的实现。轻量化后的目标检测算法在PC平台上测试了检测精度和推理时间,结果是检测精度略有降低的同时,轻量化算法的推理时间大幅降低。在嵌入式平台上的验证结果表示,本文提出的轻量化红外无人机目标检测算法能基本实现实时检测。

\section{未来工作展望}
本人在研究生期间在基于深度学习的红外无人机目标检测方面进行了一系列的探
索,既有针对PC场景的算法优化,也有针对嵌入式场景的算法轻量化和嵌入式验证,取得了一定的研究成果。但仍然存在一些需要进
一步改进的地方。后续可以从以下几个方向进行深入研究:

(1)本文建立的数据集中只包含1类无人机目标,后续研究中可以建立更完善的无人机目标数据集,丰富目标的种类。

(2)本文提出的数据增强算法,默认是对于所有的原始图像都进行增强,并且实际增强的算法参数都没有进一步实验,因此该算法还有一定的改进空间,后续可以对这个方向进行继续研究。

(3)本文的算法在NVIDIA AGX XAVIER平台上进行了验证,该平台高度封装且成本较高,因此可能并不完全符合实际检测任务的要求,因此本文的算法有待算力更低且成本较低的嵌入式设备进一步验证。此外,本文的算法嵌入式设备推理速度为30FPS左右,还有进一步提升的空间,并且本文并没有针对硬件设备进行针对性的效率优化,后续可以在这个方向上进行研究。



\end{conclusions}
